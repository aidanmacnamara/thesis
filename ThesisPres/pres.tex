\documentclass[slidescentered,compress]{beamer}

% compiled on i2 using:
% pdflatex pfizer1.tex

\usepackage{beamerthemesplit}
\usetheme{JuanLesPins}

%\usepackage{pstricks}
\usepackage{xcolor}
\usepackage{moreverb,epsfig,color,subfigure}
\usepackage{hyperref}

\usepackage{textpos}

\usepackage{graphicx}
\usepackage{longtable}
\usepackage{subfigure}
\usepackage{listings}
\usepackage{amsmath}
\usepackage{amssymb}
\usepackage{multirow}

\usepackage{bigstrut}
\usepackage{rotating}
\usepackage{colortbl}
\usepackage{array}
\usepackage{tabulary}

\beamertemplateballitem

\setbeamercolor{postit}{fg=black,bg=yellow}

\title[HLA Class I protection in HTLV-I infection]{HLA Class I protection in HTLV-I infection}
\author{Aidan MacNamara \\ \texttt{aidan.macnamara@imperial.ac.uk}}
\date{September 20, 2010}

\begin{document}

\frame{\titlepage}

\section*{Outline}
\frame{\tableofcontents}

\section{Introduction}
\subsection{The Virus}

\frame
{
\begin{columns}[c]

\begin{column}{6cm}
\frametitle{Human T-lymphotropic virus (Type 1)}
\begin{figure}
	\pgfimage[width=5cm]{images/htlv}
	\caption{Stylised Human T-lymphotropic virus}
\end{figure}
\end{column}

\begin{column}{4cm}
\begin{itemize}
\item Family Retroviridae 
\item Up to 20 million people infected  
\item Endemic: Japan, South America, Caribbean
\item Infects T-cells, mitosis / cell-cell transmission
\end{itemize}
\end{column}

\end{columns}
}

\subsection{Why study this virus?}

\frame
{
\frametitle{Individuals Infected with HTLV-I}

\begin{table}[htbp]
\centering
\begin{tabular}{lll}
95\% & $\rightarrow$ & Asymptomatic Carriers (ACs) \\ 
2-3\% & $\rightarrow$ & Chronic Inflammatory Diseases e.g.~HAM/TSP \\ 
2-3\% & $\rightarrow$ & Leukaemia (ATL) \\
\end{tabular}
\end{table}

\pause

\begin{block}{Question}
What determines these different outcomes to infection?
\end{block}

}

\frame
{
\frametitle{What determines these different outcomes to infection?}

Differences in the host immune response?

\begin{columns}[c]

\begin{column}{5cm}
\begin{itemize}
\item Certain MHC class I alleles confer protection in individuals
\item Protective - A*02 and Cw*08
\item Detrimental - B*54
\end{itemize}
\end{column}

%\pause

\begin{column}{5cm}
\begin{figure}
	\pgfimage[width=5cm]{images/mhc1}
\end{figure}

%\pause

\begin{figure}
	\pgfimage[width=5cm]{images/mhc2}
\end{figure}
\end{column}
\end{columns}
}

\frame
{
\frametitle{What determines these different outcomes to infection?}

\begin{columns}[c]

\begin{column}{4cm}
\begin{figure}
	\pgfimage[width=4cm]{images/mhc3}
	\caption{The T-cell receptor recognizes antigens bound to MHC class I}
\end{figure}
\end{column}

\pause

\begin{column}{6cm}
\begin{itemize}
\item Why do specific forms of the MHC class I genotype afford greater protection than others?
\item Differences in the cytotoxic T lymphocyte (CTL) immune response that they restrict?
\end{itemize}
\end{column}
\end{columns}

}

\subsection{Aim}

\frame
{
\frametitle{What is the role of the CD8$^+$ T cell response?}

\begin{itemize}
\item Test and improve existing epitope prediction software in order to predict HTLV-I epitopes
\item Test hypotheses about the epitope properties of protective and detrimental alleles
\item Model the CD8$^+$ T cell response in terms of its rate of lysis of infected CD4$^+$ T cells
\item Further understand the role of CTLs and NK cells in HTLV-I infection
\end{itemize}

}

\section{Method}
\subsection{Overall Scheme}

\frame
{
\frametitle{How to answer the question?}

Cohort of HTLV-I infected individuals:

\begin{itemize}
\item 202 asymptomatic carriers
\item 230 HAM/TSP patients
\item MHC class I genotype
\item Proviral load
\end{itemize}

\begin{table}[htbp]
\centering
\begin{tabular}{lllll}
HTLV-I genome & $\rightarrow$ & & & Proviral Load \\
& & Predicted Epitopes & $\approx$ & \\
MHC Class I & $\rightarrow$ & & & Disease Status
\end{tabular}
\end{table}

}

\subsection{Epitope Prediction Software}

\frame
{
\frametitle{Software Development}

\begin{columns}[c]

\begin{column}{5cm}
\begin{figure}
	\pgfimage[width=5cm]{images/pred}
	\caption{ROC curve that illustrates epitope prediction performance}
\end{figure}
\end{column}

\begin{column}{5cm}

Current Software

\begin{itemize}
\item NetMHC - predicts peptide-MHC binding
\item NetCTL - also predicts TAP and cleavage events
\item Normalise predicted binding affinities
\end{itemize}
\end{column}
\end{columns}

}

\frame
{
\frametitle{Software Development}

\begin{columns}[c]

\begin{column}{5cm}
\begin{figure}
	\pgfimage[width=5cm]{images/pred}
	\caption{ROC curve that illustrates epitope prediction performance}
\end{figure}
\end{column}

\begin{column}{5cm}

Metaserver (MacNamara et al.~PLoS Comp.~Biol.~2009)

\begin{itemize}
\item Removes normalising function	
\item Significant improvement in accuracy
\end{itemize}
\end{column}
\end{columns}

}

\frame
{
\frametitle{The Rank Method}

\scriptsize
{
\begin{table}[htp]
\centering
\begin{tabular}{c|c|c|c|c|c|c|c|}

\cline{2-8}
&& \multicolumn{2}{c|}{A*02} & \multicolumn{2}{c|}{B*54} & \multicolumn{2}{c|}{C*08} \bigstrut \\
\cline{2-8}
\multirow{3}{*}{\begin{sideways}Strong\end{sideways}} & 1 & Gag & TPKDKTKVL & Tax & LPTTLFQPA & Tax & YLYQLSPPI  \bigstrut[t] \\
& 2 & Pol & PADPKEKDL & Pro & LPVIPLDPA & Tax & LLFGYPVYV \\
& 3 & Rof & RPPPAPCLL & Env & FPFSLLVDA & Pol & ALLGEIQWV \\
& 4 & P12 & RPPPAPCLL & Pol & MPVFTLSPV & Pol & SLISHGLPV \\
& 5 & Gag & NANKECQKL & Rof & LPITMRFPA & Pol & FQPYFAFTV \\
& 6 & Gag & ANNPQQQGL & P12 & LPITMRFPA & Gag & FMQTIRLAV \\
\multirow{3}{*}{\begin{sideways}Weak\end{sideways}} & 7 & Gag & GAPPNHRPW & Pro & LPFRTTPIV & Pol & LTYDAVPTV \\
& \ldots & \ldots & \ldots & \ldots & \ldots & \ldots & \ldots \\
& 3389 & P12 & LLLFLLPPS & Tax & DNDHEPQIS & Tax & DNDHEPQIS \bigstrut[b] \\
\cline{2-8}

\end{tabular}
\end{table}

}

\pause 

\normalsize
\begin{itemize}
\item From Borghans et al.~2007
\item Defines how well an allele binds a specific protein
\item Example: A*02-Gag $\rightarrow$ 1,5,6,7...
\end{itemize}
}


\section{Results}

\subsection{Software Validation}

\frame
{
\frametitle{Software Validation for HTLV-I}

\begin{columns}[c]

\begin{column}{6cm}
\begin{figure}
	\pgfimage[width=6cm]{images/figure_1}
\end{figure}
\end{column}

\begin{column}{4cm}
\begin{itemize}
\item Predicted versus laboratory binding affinities
\item 50 HTLV-I peptides for 4 alleles
\item Strong correlation found
\end{itemize}
\end{column}

\end{columns}

}

\frame
{
\frametitle{Software Validation for HTLV-I}

\begin{block}{Conclusion}
Metaserver and Epipred predict MHC class I binding affinity to HTLV-I peptides with good accuracy
\end{block}
}


\subsection{Binding as a predictor of disease}

\frame
{
\frametitle{Do protective alleles bind different proteins?}

Do protective alleles (A*02 and Cw*08) have a different binding specificity compared to the detrimental allele (B*54)?

}


%\frame
%{
%\frametitle{The Rank Method}
%
%\scriptsize
%{
%\begin{table}[htp]
%\centering
%\begin{tabular}{c|c|c|c|c|c|c|c|}
%
%\cline{2-8}
%&& \multicolumn{2}{c|}{A*02} & \multicolumn{2}{c|}{B*54} & \multicolumn{2}{c|}{C*08} \bigstrut \\
%\cline{2-8}
%\multirow{3}{*}{\begin{sideways}Strong\end{sideways}} & 1 & \cellcolor[gray]{0.7} Gag & TPKDKTKVL & Tax & LPTTLFQPA & Tax & YLYQLSPPI  \bigstrut[t] \\
%& 2 & Pol & PADPKEKDL & Pro & LPVIPLDPA & Tax & LLFGYPVYV \\
%& 3 & Rof & RPPPAPCLL & Env & FPFSLLVDA & Pol & ALLGEIQWV \\
%& 4 & P12 & RPPPAPCLL & Pol & MPVFTLSPV & Pol & SLISHGLPV \\
%& 5 & Gag & NANKECQKL & Rof & LPITMRFPA & Pol & FQPYFAFTV \\
%& 6 & Gag & ANNPQQQGL & P12 & LPITMRFPA & Gag & FMQTIRLAV \\
%\multirow{3}{*}{\begin{sideways}Weak\end{sideways}} & 7 & Gag & GAPPNHRPW & Pro & LPFRTTPIV & Pol & LTYDAVPTV \\
%& \ldots & \ldots & \ldots & \ldots & \ldots & \ldots & \ldots \\
%& 3389 & P12 & LLLFLLPPS & Tax & DNDHEPQIS & Tax & DNDHEPQIS \bigstrut[b] \\
%\cline{2-8}
%
%\end{tabular}
%\end{table}
%
%}
%}

\frame
{
\frametitle{Do protective alleles bind different proteins?}

\begin{columns}[c]

\begin{column}{6cm}
\begin{figure}
	\pgfimage[width=6cm]{images/figure_2}
\end{figure}
\end{column}

\pause

\begin{column}{4cm}
\begin{block}{Conclusion}
Protective alleles (A*02 and Cw*08) bind HBZ significantly more strongly than a detrimental allele (B*54)
\end{block}
\end{column}

\end{columns}
}

\frame
{
\frametitle{Do ACs and HAM/TSP patients bind different proteins?}
\begin{table}[htbp]
\centering
\begin{tabular}{lll}
202 AC individuals & & Predicted epitopes \\
HLA class I type & $\longrightarrow$ & from each protein \\
\\
& & \bf{Vs.} \\
\\
230 HAM/TSP individuals & & Predicted epitopes \\
HLA class I type & $\longrightarrow$ & from each protein \\
\end{tabular}
\end{table}
}

\frame
{
\frametitle{Do ACs and HAM/TSP patients bind different proteins?}

\begin{columns}[c]

\begin{column}{6cm}
\begin{figure}
	\pgfimage[width=5cm]{images/figure_3}
\end{figure}
\end{column}

\pause

\begin{column}{5cm}
\begin{block}{Conclusion}
Asymptomatic carriers possess HLA molecules that strongly bind HBZ
\end{block}
\end{column}

\end{columns}
}


\frame
{
\frametitle{Why is binding HBZ protective?}

\begin{block}{Hypothesis}
Binding HBZ is associated with a reduced proviral load
\end{block}

}

\frame
{
\frametitle{Why is binding HBZ protective?}

\begin{figure}
	\pgfimage[width=8cm]{images/figure_4}
\end{figure}

}

\frame
{
\frametitle{Why is binding HBZ protective?}

\begin{block}{Conclusion}
Binding HBZ is associated with a reduced proviral load
\end{block}

}



\frame
{
\frametitle{Is There a Link Between risk and proviral load with other proteins?}
\begin{columns}[c]

\begin{column}{5cm}
\begin{figure}
	\pgfimage[width=5cm]{images/figure_5}
\end{figure}
\end{column}

\begin{column}{5cm}
For each protein 2 hypotheses were tested:
\begin{itemize}
\item Binding the protein is associated with reduced HAM/TSP risk
\item Binding the protein is associated with reduced proviral load
\end{itemize}
\end{column}

\end{columns}
}

\frame
{
\frametitle{Is There a Link Between risk and proviral load with other proteins?}

\begin{block}{Conclusion}
\begin{itemize}
\pause
\item ACs possess alleles that bind more strongly to proteins associated with a reduced proviral load
\pause
\item This is not related to immunogenicity
\pause
\item Strong evidence of a protective CTL response (MacNamara et al. PLoS Pathogens, 2010)
\end{itemize}
\end{block}

}


%\frame
%{
%\frametitle{Binding specificity is a better predictor of load}
%
%\scriptsize
%{
%\begin{table}[htp]
%\centering
%
%\begin{tabular}{|l|l|l|l|l|l|l|}
%\hline
%& \multicolumn{3}{c|}{Binding (A and B only)} & \multicolumn{3}{c|}{Genotype (A and B only)} \bigstrut \\
%\hline
%\multirow{3}{*}{AC Proviral Load} & HBZ & 0.001 & *** & \multirow{2}{*}{A*02} & \multirow{2}{*}{0.01} & \multirow{2}{*}{**} \bigstrut[t] \\
%& Pro & 0.013 & * & & & \bigstrut[b] \\
%\cline{2-7}
%& \multicolumn{3}{c|}{$R^2 = 0.054$} & \multicolumn{3}{c|}{$R^2 = 0.034$} \bigstrut \\
%\hline
%\multirow{2}{*}{HAM/TSP Proviral Load} & HBZ & 0.017 & * & B*54 & 0.019 & * \bigstrut \\
%\cline{2-7}
%& \multicolumn{3}{c|}{$R^2 = 0.026$} & \multicolumn{3}{c|}{$R^2 = 0.025$} \bigstrut \\
%\hline
%\end{tabular}	
%\end{table}
%}
%
%}



\subsection{Theory to experiment}

\frame
{
\frametitle{Can we detect HBZ-specific CTL?}

\begin{columns}[c]

\begin{column}{6cm}
\begin{figure}
\pgfimage[width=6cm]{images/exper}
\caption{IFN-$\gamma$ ELISpot that demonstrates a CTL response against HBZ from a HTLV infected patient (compared to control, left)}
\end{figure}
\end{column}

\begin{column}{4cm}
\begin{itemize}
\item Detection of HBZ-specific CD8$^+$ T cells (IFN$\gamma$, CD107a)
\item Demonstration of HBZ-specific lysis (B-LCL, naturally infected)
\end{itemize}
\end{column}

\end{columns}

}

\frame
{
\frametitle{Why HBZ?}

\begin{table}[htbp]
\centering
\renewcommand{\arraystretch}{1.2}
\begin{tabulary}{10cm}{|L|c|L|}
\cline{1-1}
\cline{3-3}
HBZ inhibits expression of other HTLV-I genes & & HBZ expression drives infected cell proliferation \\
$\Rightarrow$Protects cell from immune response & & \\ 
\cline{1-1}
\cline{3-3}
\multicolumn{3}{c}{$\Downarrow$} \\ 
\end{tabulary}
\end{table}

\begin{block}{Hypothesis}
HBZ expressing cells have a survival advantage and blocking this pathway is important
\end{block}

}

%%%%%%%%%%%%%%%%%%%%%%%%%%%%%%%%%%%%%%%%%%%%%%%%%%%%%%%%%%%%


\subsection{The role of CTLs and NK cells}

\frame
{
\frametitle{The rate of lysis of infected cells}

\begin{columns}[l]

\begin{column}{6cm}

\begin{figure}
\pgfimage[width=6cm]{images/assay}
\end{figure}

\begin{equation*}
\frac{dy}{dt} = c - \epsilon y z
\end{equation*}

\end{column}

\begin{column}{4cm}
\begin{itemize}
\item Flow Cytometry detects level of Tax expression
\item CD8$^+$ T cells added at increasing concentrations
\item The lysis rate measured using model
\end{itemize}
\end{column}
\end{columns}

}

\frame
{
\frametitle{The rate of lysis of infected cells}

\begin{columns}[c]

\begin{column}{5cm}
\begin{block}{Time course of tax expression}
Target CD4$^+$ cells divided into Tax$^{high}$ and Tax$^{low}$ 
\end{block}

\begin{figure}
\pgfimage[width=4cm]{images/figure_timecourse_hap}
\end{figure}

%\begin{equation*}
%\begin{array}{ccl}
%\frac{dy}{dt} &=& c_1 - c_2y  \\
%\\
%\frac{dw}{dt} &=& c_{2}y \\
%\end{array}
%\end{equation*}

\end{column}

\pause

\begin{column}{5cm}

\begin{block}{Lysis of CD4$^+$ cells}
Rate of lysis measured for Tax$^{high}$ and Tax$^{low}$
\end{block}

\begin{figure}
\pgfimage[width=4cm]{images/figure_lysis_hap_rep_1}
\end{figure}

\end{column}

\end{columns}

}

\frame
{
\frametitle{The rate of lysis of infected cells}

\begin{columns}[c]

\begin{column}{5cm}
\pgfimage[width=5cm]{images/pairedEpsilon}
\end{column}

\begin{column}{5cm}
\begin{itemize}
\item Target CD4$^+$ cells are killed quicker when expressing higher levels of Tax
\item MacNamara et al. J.~Immunol. 2009
\end{itemize}
\end{column}

\end{columns}
}

%\frame
%{
%\frametitle{The role of KIR-HLA interactions}
%\begin{columns}[c]
%\begin{column}{5cm}
%\begin{figure}
%\pgfimage[width=2.5cm]{images/figureActivAC}%
%\hspace{0cm}%
%\pgfimage[width=2.5cm]{images/figureActivHAM} \\
%\pgfimage[width=2.5cm]{images/figureInhibAC}%
%\hspace{0cm}%
%\pgfimage[width=2.5cm]{images/figureInhibHAM} \\
%\end{figure}
%\end{column}
%\begin{column}{5cm}
%\begin{itemize}
%\item KIR:HLA interactions - shown to be protective in some diseases
%\item No significant association with disease status or proviral load in HTLV-I
%\end{itemize}
%\end{column}
%\end{columns}
%}

\section{Conclusion}

\begin{enumerate}

\frame
{
\frametitle{Summary}

\footnotesize
{

\begin{columns}[c]

\begin{column}{5cm}
\pgfimage[width=2.5cm]{images/pred}
\item Developed accurate epitope prediction software
\pgfimage[width=2.5cm]{images/figure_1_conclusion}
\item Validated software experimentally for 200 HTLV-I peptides
\end{column}

\begin{column}{5cm}
\pgfimage[width=3cm]{images/figure_2}
\item Protective alleles bind HBZ more strongly than a detrimental allele 
\pgfimage[width=2.5cm]{images/figure_3}
\item ACs have HLA alleles which are specific for HBZ
\end{column}

\end{columns}
}
}

\frame
{
\frametitle{Summary}

\footnotesize
{

\begin{columns}[c]

\begin{column}{5cm}
\pgfimage[width=2.5cm]{images/figure_4}
\item Binding HBZ is associated with a reduced proviral load
\pgfimage[width=2.5cm]{images/figure_5}
\item Proteins that are preferentially targeted by ACs are those associated with a greater reduction in load when bound
\end{column}

\begin{column}{5cm}
\pgfimage[width=3cm]{images/exper}
\item HBZ-specific responses exist 
\pgfimage[width=3cm]{images/figure_lysis_hap_rep_1}
\item Antigen expression levels affect lysis rate and no KIR:HLA associations were found
\end{column}

\end{columns}
}
}

\end{enumerate}

\section*{Acknowledgements}

\frame
{

Primary supervisor: Dr.~B Asquith

Secondary supervisor: Prof.~Charles Bangham

\vspace{0.5cm}

Collaborators \& Data:
\begin{itemize}
\item Aileen Rowan, Silva Youshya
\item Ulrich Kadolsky 
\end{itemize}

\vspace{0.5cm}

\begin{figure}
\begin{center}
\pgfimage[width=5cm]{images/wellcome}
\end{center}
\end{figure}

}

%%%%%%%%%%%%%%%%%%%%%%%%%%%%%%%%%%%%%%%%%%%%%%%%%%%%%%%%%%%%%%%%%%%%%%%%%%%%%%%%%%%%%%%%%%%%%%%%%%%%
%%%%% END OF %%%%%
%%%%%%%%%%%%%%%%%%%%%%%%%%%%%%%%%%%%%%%%%%%%%%%%%%%%%%%%%%%%%%%%%%%%%%%%%%%%%%%%%%%%%%%%%%%%%%%%%%%%


\end{document}